\section{Costs}\label{sec:costs}

As mentioned in chapter \ref{sec:tooling} there are two popular solutions for
Continuous Integration systems.  We described them from a technical view and
then explained self-hosted Continuous Integration systems in chapter
\ref{sec:tooling-self-hosted} and cloud-based Continuous Integration systems in
chapter \ref{sec:tooling-cloud-based}.  In this chapter, we compare both
Continuous Integration solutions on the basis of economic factors.  We can
categorize the costs of a Continuous Integration system in following categories:

\begin{itemize} \item License costs 
    \item Maintenance costs 
    \item Hardware costs
\end{itemize}

\subsection{License costs}

The license costs for self-hosted Continuous Integration solutions are mostly
nil, because they are often open source and it does not matter if you use them
for private or public repositories.  Cloud-based Continuous Integration systems
on the other hand often have a monthly subscription pricing model for private
repositories, but a completely free pricing model for public open-source
projects.

\subsection{Maintenance costs}

Both Continuous Integration solutions have maintenance costs. Cloud-based
Continuous Integration systems are much easier to configure than self-hosted
Continuous Integration systems.

\subsection{Hardware costs}

Another benefit of using cloud-based Continuous
Integration systems is that the users of the system do not have to buy hardware
to use it, because it is hosted in the cloud.

\subsection{Cost Conclusion}

Self-hosted Continuous Integration systems are more extensible than cloud-based
Continuous Integration systems, but cloud-based Continuous Integration systems
are easier to configure and maintenance. There is no best solution for
everything, both have there advantages and disadvantages.
