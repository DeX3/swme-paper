\section{Costs}\label{sec:costs}

As mentioned in chapter \ref{sec:tooling} there are two popular solutions for continious-integration systems.
 We descriped them from a technical view and and explained self-hosted continious-integration systems in chapter \ref{sec:tooling-self-hosted} and cloud-based continious-integration systems in chapter \ref{sec:tooling-cloud-based}.
 In this chapter we compared both continious-integration solutions on the basis of economic factors.
 We can categorize the costs of a continious-integration system in following categories:

\begin{itemize} 
    \item License costs 

    \item Maintenance costs 

    \item Hardware costs

\end{itemize}


\subsection{License costs}

The license costs for self-hosted continious-integration solutions are mostly none, because they are often open source
and it does not matter if you use them for private or public repositories. 
Cloud-based continious-integration systems on the other hand often have a monthly subsciption
pricing model for private repositories, but a completly free pricing model for 
public open-source projects.

\subsection{Maintenance costs}

Both continious-integration solutions have maintenance costs. Cloud-based continious-integration systems are much easier to configure than self-hosted continious-integration systems.

\subsection{Hardware costs}
Another benefit of using cloud-based continious-integration systems is that the users of the system
do not have to buy hardware to use it, because it is hosted in the cloud.

\subsection{Cost Conclusion}

Self-hosted continious-integration systems are more extensible than cloud-based 
continious-integration systems, but cloud-based continious-integration 
systems are easier to configure and maintenance. There is no best soultion 
for everything, both have there advantages and disadvanteges.
