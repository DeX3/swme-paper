\section{Introduction}\label{sec:introduction}

In today's fast-moving software world, no large software project is developed by
one single person. Team coordination becomes a challenging factor in software
development, as success rates drop with increasing team sizes
\cite{ambler:2010}. Another central pillar of modern software development is
testing. While proper software testing clearly boosts software quality, it is
incredibly expensive \cite{dustin:1999}.\\

Continuous Integration is an approach to addresses these problems by
\begin{enumerate}[label=(\alph*)]
    \item Applying small incremental changes over time to a central source code
       repository
    \item Employing an automated testing strategy
\end{enumerate}

and strives to improve team efficiency all together.

\subsection{History}\label{sec:history}

When Continuous Integration first became popular, it was tightly coupled to many
concepts practiced in Extreme Programming. In this software development
methodology, (typically small) development teams use various coding-focused
techniques in order to develop software faster.

