\section{Conclusions}\label{sec:conclusions}

The biggest benefit of Continuous Integration is reduced risk. Long and big
projects are hard to oversee. With Continuous Integration there is no long
integration phase. There is no blind spot where you don't know how long it will
take to get a working build. At all times, you know what works and what does
not.  You always know when a Bug appears and when it disappears. Bugs hinder
your progress and with Continuous Integration they are more manageable. Of
course with Continuous Integration alone you will not magically get rid of all
bugs. It just makes them a lot easier to find. Because there are only small
changes to check, bugs are far more fixable. Also that changed code is just
fresh in memory of the developer, so the fix should be easier to find.  It
prevents the cumulation of multiple bugs, which make a single bug even harder to
fix \cite{fowler:2006}.\\

According to \cite{weiss:2013}, additional benefits of Continuous Integration
are:

\begin{itemize}
    \item Eradication of manual FTP deployment
    \item Prevention \& reduction of production \& staging errors
    \item Generation of analysis \& reporting on the health of the code base
\end{itemize}

In business terms, the value of Continuous Integration is:
\begin{itemize}
    \item Reducing risk
    \item Reducing overheads across the development \& deployment process
    \item Enhancing the reputation of the company by providing Quality Assurance
\end{itemize}


