\section{Git-workflows}

Because the way we use our version control management influences the way
Continuous Integration works, we want to depict different approaches to version
control management \cite{atlassian:2013}.

\subsection{Centralized Workflow}\label{sec:centralized-workflow}

This workflow does not require any other branches beside \textit{master}. It
uses a central repository to serve as a single point-of-entry for all changes to
the project. When a developer wants to commit his changes, he pushes them to
the master branch. In case of a conflict, the developer has to fetch the updates
of the other developers first. After that, the developer is up-to-date and can
now commit his changes, this style results in a perfect linear history of
changes. A rebase prevents big merge-commits, it tries to merge commit by
commit, which results in less conflicts and an easier way to see, when bugs got
introduced. This usage of Git resembles the classic SVN-style the most and
makes transitioning between those two systems the easiest. \\

\subsection{Feature Branch Workflow}

The general idea of this workflow is to use a branch for every feature that
should be implemented. This means every developer can work undisturbed on a
particular feature. This way the master branch will never contain broken code,
which is a huge advantage for Continuous Integration. Also, pull requests are a
way to monitor branches and to discuss them. They make it easy for other
developers to comment on branches.\\

There is still a central repository in use but developers now don't commit
directly to the master branch. Whenever a developer wants to start a new
feature, he creates a so called feature-branch. Now they work on this branch as
usual, editing, staging and committing. The branches should have descriptive
names for the feature they present. The purpose of the branch should be clear
and focused. Branches can also be pushed to the main repository, so they
can be shared between developers without pushing it on to the 'official code'.
This way, also the local commits are easily backed up.\\

This way after a feature is completed it will get into the master branch but not
without a pull request. This pull request 'asks' another developer to be merged
with the master branch. It gives that developer a chance to review the changes
before they become a part of the main codebase. This does not have to be only a
code review but is the chance to talk about the committed code. It is also
possible this way to ask for help. If one developer is stuck he can issue a pull
request and other developer get a notification automatically. If a pull
request is accepted, the same work has to be done this time as in the previous
workflow. The local master has to be synchronized with the upstream master then
the changes of the feature branch are merged into the local master and then this
updated master gets pushed back into the central repository.

\subsection{Gitflow Workflow}\label{sec:git-flow}

The Gitflow Workflow is an arguably stricter version of the Feature Branch
Workflow. The branching model is thus more strict and resolves around the
project release. It does not add new commands or concepts, but assigns different
purposes to branches. In addition to the known feature branches there are now
also branches for preparing, maintaining, and recording releases. Those
different roles define how those branches interact. Of course the same benefits
of feature Branch Workflow apply.\\

Still every developer works locally and pushes branches to the central
repository, the only difference is the branch structure of the project. This
workflow uses a develop branch parallel to the master branch. The master branch
stores the official release history while the develop branch serves an
integration branch for features. All commits to the master branch should be
tagged with a version number. Feature branches now don't interact directly with
the master branch. They branch off the develop branch and also get merged back
to the develop branch, when it's complete. This part behaves like the Feature
Branch Workflow, but it does not stop here. Once develop has enough features for
the next release, you fork a release branch off of develop. This dedicated
branch for releases can be polished and get bugfixes once this branch is ready
to ship, it gets merged back to master and develop. For hotfixes or maintenance
it is ok to branch directly from master, to quickly patch production releases.\\

\subsection{Forking Workflow}

This workflow is inherently different than the other workflows. The Forking
Workflow gives every developer also an extra server-side repository to commit
to. This means every developer has now two repositories. A private local one and
a public server-side one. This way the contributions can be integrated without
the need to give every developer access to a single central repository. The
project manager can accept the features and integrate them without giving the
developer write access to the official codebase. This distributed workflow
provides a flexible way for large teams. Now they are able to collaborate in a
secure way. This is an ideal model for an open-source project.\\

The main difference here is just how those branches are shared between
developers. For Git all those server-side repositories are the same and the
official one is just the public repository of the project maintainer. Still
every developer should use his own branches for different features and push them
to his own server-side repository. Those branches will be pulled into the local
repository of another developer so they can finally end into the official
server-side repository.

