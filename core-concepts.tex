\section{Core concepts}\label{sec:core-concepts}

In this section, we discuss the integral core concepts of Continuous Integration
in order to give the reader a good understanding of its principles.

\subsection{Automation}\label{sec:automation}

It would not be possible to facilitate Continuous Integration without a high
degree of automation present in the building process. The CI-server has to be
able to perform each task (building, running tests, analysis, deployment) at
least mostly automatically, or the concept would not be feasible. It makes sense
cost-wise that the build-server uses the same system developers use locally to
build the product, thus Continuous Integration has to be taken into account when
first designing the project's development architecture.\\

\subsubsection{Triggers}\label{sec:triggers}

Builds in Continuous integration systems need to be triggered by some kind of
event. The trigger usually falls into one or more of the following categories:

\begin{itemize}
    \item Manually
    \item Timed
    \item On change
\end{itemize}

The simplest form of trigger is a manual one. It can be used as a fallback when
other triggers are unavailable and can also be utilized for testing the
build-system itself. Continuous Integration systems usually provide this
functionality via a user interface such as a website.\\

For timed builds, a simple time-frame is configured for the build to happen -
e.g. daily, hourly, etc. The Continuous Integration system can employ its own
timing mechanism or rely on frameworks provided by the operating system (e.g.
cronjobs).\\

The most useful type of trigger is a change-based trigger. Here, the Continuous
Integration system listens for changes on the project's source code repository
and only builds the project when an actual change has occurred. This trigger
enables developers to get almost instant feedback for whether their change has
broken the build or not.

The listening itself can be accomplished by a simple polling mechanism, however
the cleanest approach is based on VCS-based hooks. Here, the version-control
system provides a possibility to execute arbitrary commands on certain events,
such as a commit/merge on a specific branch.\\

GIT for example supports both client-side and server-side hooks. Client-side
hooks can be used to do some pre-push checks, but they have to be configured
individually for each developer. The really useful type of git-hook in terms of
Continuous Integration are server-side hooks of which GIT provides 3 types
\cite{chacon:2009}:

\begin{itemize}
    \item pre-receive: Run when handling a push from a client
    \item update: Similar to pre-receive, but run for each branch
    \item post-receive: Runs after the push process has completed. This is
        typically used for triggering a CI-based build.
\end{itemize}

Similar hooking mechanisms are available in other version control systems, such
as
SVN\footnote{\url{http://svnbook.red-bean.com/nightly/en/svn.reposadmin.create.html}}
or Mercurial\footnote{\url{http://mercurial.selenic.com/wiki/Hook}}.\\

Triggers can of course be combined to provide multiple trigger points.

\subsection{Build types}\label{sec:build-types}

Continuous Integration systems usually go beyond the capabilities of a simple
build-server. While running tests is also a main-task for a CI-system, they also
have the capabilities to measure code metrics or perform actual deployment.

\subsubsection{Builds}\label{sec:builds}

The most common type of build is essentially what "build" used to refer to for
local machines, only for a central Continuous Integration server. Depending on
the type of applications, files are compiled, linked, minified, obfuscated and
more. In short, all steps necessary to build a functional production-ready
instance of the application are executed in order. The result of a successful
build is a deployable application. Creating an actual build is in the majority of cases a necessary precondition to
execute any other build type.\\

A big build can take time, so Continuous Integration systems have to be smart
about it and for example only recompile source files that actually have changed.

The build process on the CI-server is ideally largely identical to the build
process developers use on their local machines. This ensures coherent error
reporting (avoiding questions like "why did it work for me, then?"). The
tool-landscape for build automation has grown immensely in the past years, so
that a large toolset is available to use. Various frameworks and programming
languages have spawned different incarnations of build systems, such as GNU
Make\footnote{\url{http://www.gnu.org/software/make/}} (C, C++, general
purpose), Ant\footnote{\url{http://ant.apache.org/}},
Maven\footnote{\url{https://maven.apache.org/}} or
Gradle\footnote{\url{https://gradle.org/}} (Java), to web-specialized tools such as
Grunt\footnote{\url{http://gruntjs.com/}} or
Gulp\footnote{\url{http://gulpjs.com/}}.\\

These systems all follow a basic workflow. You describe to the tool how you want
your project to be built (in a "Makefile", "Gruntfile", ...), and the tool will
then build your project for you. Typically these tools are smart enough to only
recompile files that have changed.\\

Additionally, some tools also incorporate dependency management. In this case
the tools provide a way to download missing libraries or frameworks directly on
build. This is actually a great feature to have, because with dependency
management, the need to have all libraries checked into version control
vanishes. This contributes to smaller commits and fewer merge conflicts (see
\ref{sec:motivation}), because
when a library is updated, ideally, the only affected line is the one that tells
the build system what version to use.\\

However, this approach also has a drawback. Since the repository does not hold
all the code necessary to build a functional application, the build is dependent
on external services. Examples for these services are the Maven
repository\footnote{\url{http://mvnrepository.com/}} for Java libraries or the
npm-registry\footnote{\url{https://www.npmjs.com/}} for node.js packages. This
means, that if the external server is not available for whatever reason, no
builds are possible during the downtime.

A possible workaround would be to still manage the libraries via dependency
management systems, but also check the downloaded libraries into version
control. While this reintroduces the problem of large commits, it solves the
issue of external service dependency. Another solution would be to host your own
clone of the external service. While most of the tools basically support this
\footnote{\url{https://docs.npmjs.com/misc/registry\#can-i-run-my-own-private-registry}}
\footnote{\url{https://outofsingularity.wordpress.com/2011/05/15/create-custom-maven-repository-to-store-artifacts-unavailable-in-general-maven-repositories/}},
it is a complex and expensive approach.

\subsubsection{Testing}\label{sec:testing}

It clearly makes sense to include automatic tests into the Continuous
Integration process. With the CI-server performing tests on the built
application, it becomes not only a baseline indicator for whether the code
compiles, but also for whether the code actually performs its tasks. Of course
tests will likely never be able to verify the application to a hundred percent,
but they can at least give an indication.

However, setting up proper testing in a CI-environment can be challenging. The
test environment has to be self-contained. Most complex applications depend on
other processes to function (e.g. databases, message brokers, communication
pipelines, etc.). In order to be able to test the application, these services
must also be bootstrapped and configured properly for testing. Ideally, the
services are reset to a clean state before each test run. Indeed, it is
considered a best practice to run the tests in a clone of an actual production
environment.\\

In order to keep the tests up to date with the continuously developed
application, it is essential, that the tests themselves are extended steadily.

Automated tests are usually done in a unit-testing framework (e.g. JUnit for
Java). These frameworks usually provide means of exporting test-result data so
that it can be picked up by the CI-system (and used for reporting).

\paragraph{Code coverage}
During the test runs, it can be useful to collect
additional meta-information about the test run. Code-coverage tools provide a
way of easily determining the breadth of the tests. Depending on the used
language/framework code coverage can very easily be integrated into the testing
process and thus provide additional data for reporting \cite{kajo-mece:2012}.

\subsubsection{Static code analysis}\label{sec:static-code-analysis}

Alongside testing, can also collect static information about checked in source
code. Common use-cases are as follows \cite{bommer:2008}, \cite{sneed:2010}:

\begin{itemize}
    \item Linting
    \item Style-checking
    \item Advanced checks for:
    \begin{itemize}
        \item Race conditions
        \item Buffer overflows
        \item Memory leaks
    \end{itemize}
    \item Architectural metrics
\end{itemize}

These static code analysis tools can further provide more information to report
to the development team, which contributes to getting a better overview over the
overall code quality present in the project source code.

\subsubsection{Deployment}\label{sec:deployment}

It is also possible to incorporate automatic deployment into a Continuous
Integration build. In this case, successful builds are automatically injected
into a production-ready environment. This setup is commonly referred to as
"Continuous Delivery".

\paragraph{Migrations}\label{sec:migrations}

Some of the challenges of Continuous Delivery are similar to the ones faced for
setting up proper Testing (see \ref{sec:testing}). Often, new versions of the
project require changes in an underlying data-model. 

For SQL-based solutions, this issue is addressed with migrations. A migration is
some form of database-script (often in SQL or some domain-specific-language)
that will update the existing database schema to the correct version while
taking care of not losing any existing data.

For NOSQL solutions like MongoDB, this issue can be solved similarly, however
since these data stores typically don't enforce a consistent database schema,
this can also be handled directly in the application. For this approach, the
application has to handle this "fuzzy" concept of a schema on its own, which can
clutter up the code.

\subsection{Source control}\label{sec:source-control}

The version control system is the backbone and therefore a pivotal element of
Continuous Integration. It supplies the central baseline repository that the
CI-server will operate upon. Additionally, it provides hooks so that the server
can trigger event-based builds (see \ref{sec:triggers}).\\

Continuous Integration can be practiced with a multitude of version control
systems. However, as branching is a central concept in Continuous Integration,
distributed version control systems like GIT - where branching is a cheap local
operation - are ideal for this kind of scenario.

A well thought-out branching model can help streamline the development process
and therefore boost productivity in a Continuous Integration process. Various
development teams have developed their own branching models for this situation,
one such example being git-flow \cite{driessen:2010}.\todo{Explain git-flow?}\\

\subsection{Reporting}\label{sec:reporting}


