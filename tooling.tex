\section{Tooling}\label{sec:tooling}

There exists many good solutions for hosting a continious-integration System. 
This chapter handles the most popular ones.
We categorized the continious-integration solution in 2 parts. 
\begin{itemize} 
    \item self-hosted continious-integration systems
    \item cloud-based continious-integration systems 
\end{itemize}

The first part is about continious-integration-Systems, which are hosted on a own private server. 
The second part of this chapter is about continious-integration-Systems, which are hosted in the cloud, which are becoming increasingly popular in the open-source community.

\subsection{Self-hosted}\label{sec:tooling-self-hosted}

Self-hosted continious-integration systems are older than cloud-based continious-integration systems.
The software for the continious-integration-System runs localy on a private server 
and the users can reach and use this system over the network. 
The most popular solutions for a self-hosted continious-integration are:
\begin{itemize} 
    \item Jenkins(former Hudson) \footnote{\url{https://jenkins-ci.org/}}
    \item TeamCity \footnote{\url{https://www.jetbrains.com/teamcity/}}
    \item CruiseControl \footnote{\url{http://cruisecontrol.sourceforge.net/}}
\end{itemize}

\subsubsection{Jenkins}
The by far most remarkable and famoust one of this continious-integration-systems is Jenkins. 
Jenkins is written in the Java programming language and completly open source released under the MIT License.
Many top companies like netflix \footnote{\url{https://www.netflix.com}} and open source projects
are using Jenkins as their continious-integration-system.
Jenkins runs in a servlet container on an application server like GlassFish \footnote{\url{https://glassfish.java.net/}}
or Apache Tomcat \footnote{\url{http://tomcat.apache.org/}} \\
To install Jenkins you can simple download the binary and install it on the operating-system.\\
Jenkis can run on following operating systems: 
There are many version-control tools included in Jenkins but the most relevant ones are git and svn.
The most powerfull feature of Jenkins is the extensibility of this continious-integration-System 
with plugins.\\

\begin{itemize} 
        \item Windows
        \item Ubuntu/Debian
        \item Red Hat/Fedora/CentOS
        \item Mac OS X
        \item openSUSE
        \item FreeBSD
        \item OpenBSD
        \item Gentoo
\end{itemize}

There are more than 1000 official plugins available for Jenkins to customize the continious-integration-System.  
One example of the extensibility of Jenkins is to distribute one or more  Jobs on many 
different machines over the network. That can reduce the running time dramatically and 
is a realy good and effective solution if the system has a big Unit-Test-Suite or GUI-Tests 
like selenium \footnote{\url{http://www.seleniumhq.org/}}, which opens a browser every time 
a test is started. The different machines are called nodes in Jenkins.

\subsubsection{TeamCity}

TeamCity is also a self-hosted continious-integration system build by JetBrains. Companies like
Apple \footnote{\url{https://www.apple.com}} and Siemens \footnote{\url{http://www.siemens.com/}} useTeamCity as their continous-integration system.\\
Teamcity has a different pricing model than Jenkins. 
It is free forever but has following limitations:

\begin{itemize} 
        \item 20 build configurations
        \item 3 build agents
\end{itemize}

Build agents are similar to Jenkins nodes and a powerfull feature of TeamCity. If you want more than 3 build agents for your jobs you have to buy the Enterprise edition.

\subsection{Cloud-based}\label{sec:tooling-cloud-based}

Cloud-based continious-integration systems are continious-integration-systems which are hosted in the cloud.
They are easier to configure than self-hosted continious-integration-systems but have their limitations in the extensibility.
They often have a monthly subscription pricing model for private version-control repositories 
and a free model for public and open-source version-control repositories.
Thats why they are becoming increasingly popular in the open-source community.
The most popular cloud-based continious-integrations are:
\begin{itemize} 
    \item Travis CI \footnote{\url{https://travis-ci.org/}}

    \item Codeship \footnote{\url{https://codeship.com/}}

    \item Drone \footnote{\url{https://drone.io/}}

\end{itemize}

\subsubsection{Travis CI}
The most popular cloud-based continious-integration system at the moment is Travis CI.
Travis CI is written in the Ruby programming language and became very popular because of the easy integration with the famous git-repository hosting service GitHub.\footnote{\url{https://github.com/}}\\
Over 200.000 open-source projects like nodeJs,\footnote{\url{https://nodejs.org/}}
 Ruby on Rails \footnote{\url{http://rubyonrails.org/}}
and emberJs \footnote{\url{http://emberjs.com/}}
and companies like facebook,\footnote{\url{https://www.facebook.com/}}
 mozilla,\footnote{\url{https://www.mozilla.org}}
 twitter \footnote{\url{https://twitter.com/}}
are using Travis Ci as their continious-integration system.\\
The configuration for Travis CI is done in one simple YAML file with the name travis.yml 
which must be located in the root directory of a GitHub repository.
There are two possible workflows how you can integrate Travis Ci:
\begin{itemize} 
    \item Branch build flow
    \item Pull request build flow
\end{itemize}
\paragraph{Branch build flow}
In the branch build flow GitHub automaticly triggers a Travis Ci build everytime you 
push a commit to your remote GitHub repository.\\
If the build is OK and all the tests succeed Travis Ci is able to deploy the 
code on a Paas like heroku.
\paragraph{Pull request build flow}
In the pull request build flow everytime you create a pull request on GitHub a Travis Ci build is triggered.

\subsubsection{Drone}

Drone is developed by Google and like Travis CI a cloud-based continious-integration system.\\
Drone integrates seamlessly with the two popular version-control hosting services 
Github and Bitbucket \footnote{\url{https://bitbucket.org/}}
as well as Platform-as-a-service providers such as Heroku, Dotcloud \footnote{\url{https://www.dotcloud.com/}},
Google AppEngine  \footnote{\url{https://cloud.google.com/appengine/}}.

