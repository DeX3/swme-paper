\section{Tooling}\label{sec:tooling}

There exist a number of good solutions for hosting a Continuous Integration
system. This chapter handles the most popular ones. We categorize the
Continuous Integration solution into two classes. 

\begin{itemize} 
    \item self-hosted Continuous Integration systems
    \item cloud-based Continuous Integration systems 
\end{itemize}

The first part is about Continuous Integration systems which are hosted on a
own private server. The second part of this chapter is about Continuous
Integration systems which are hosted in the cloud, which are becoming
increasingly popular in the open-source community.

\subsection{Self-hosted}\label{sec:tooling-self-hosted}

Self-hosted Continuous Integration systems are older than cloud-based Continuous
Integration systems. The Continuous Integration application runs locally on a
private server and the users can reach and use this system over the network. The
most popular solutions for a self-hosted Continuous Integration are:

\begin{itemize} 
    \item Jenkins(formerly known as Hudson)
        \footnote{\url{https://jenkins-ci.org/}}
    \item TeamCity \footnote{\url{https://www.jetbrains.com/teamcity/}}
    \item CruiseControl \footnote{\url{http://cruisecontrol.sourceforge.net/}}
\end{itemize}

\subsubsection{Jenkins}

The by far most remarkable and famous one of these Continuous Integration
systems is Jenkins. Jenkins is written in the Java programming language and
completely open source released under the MIT License. Many top companies like
Netflix \footnote{\url{https://www.netflix.com}} and various open source
projects are using Jenkins as their Continuous Integration system. Jenkins runs
in a servlet container inside of an application server like GlassFish
\footnote{\url{https://glassfish.java.net/}} or Apache Tomcat
\footnote{\url{http://tomcat.apache.org/}}. To install Jenkins you can simple
download an installer-binary and use it to install it on the operating-system.\\

Jenkins can run on following operating systems: 

\begin{itemize} 
        \item Windows
        \item Ubuntu/Debian
        \item Red Hat/Fedora/CentOS
        \item Mac OS X
        \item openSUSE
        \item FreeBSD
        \item OpenBSD
        \item Gentoo
\end{itemize}

Jenkins supports a host of version-control systems, the most relevant ones being
git and svn. The most powerful feature of Jenkins is its extensibility with
plugins.\\

There are more than 1000 official plugins available for Jenkins to customize the
Continuous Integration system. One example of the extensibility of Jenkins is
the ability to to delegate one or more Jobs to other slave-machines on the
network. This can reduce the time of builds dramatically and is a really good
and effective solution if the system has a big Unit-Test-Suite or GUI-Tests like
Selenium \footnote{\url{http://www.seleniumhq.org/}}, which opens a browser
every time a test is started. The different machines are called nodes in
Jenkins.

\subsubsection{TeamCity}

TeamCity is another self-hosted Continuous Integration system built by
JetBrains. Companies like Apple \footnote{\url{https://www.apple.com}} and
Siemens \footnote{\url{http://www.siemens.com/}} use TeamCity as their
Continuous Integration system.\\

TeamCity has a different pricing model than Jenkins. It is generally free, but
has the following limitations/restrictions:

\begin{itemize} 
        \item no more than 20 build configurations
        \item no more than 3 build agents
\end{itemize}

Build agents are similar to Jenkins' nodes and a powerful feature of TeamCity.
If more than 3 build agents are desired to work on build jobs, TeamCity provides
a (fee-based) Enterprise edition.

\subsection{Cloud-based}\label{sec:tooling-cloud-based}

Cloud-based Continuous Integration systems are Continuous Integration systems,
that - as their name suggests - are hosted in the cloud. They are easier to
configure than self-hosted Continuous Integration systems but have their
limitations in their extensibility. They often have a monthly subscription
pricing model for private version-control repositories and a free model for
public and open-source version-control repositories. That is why they are
becoming increasingly popular in the open-source community. The most popular
cloud-based Continuous Integrations are: 

\begin{itemize} 
    \item Travis CI \footnote{\url{https://travis-ci.org/}}
    \item Codeship \footnote{\url{https://codeship.com/}}
    \item Drone \footnote{\url{https://drone.io/}}
\end{itemize}

\subsubsection{Travis CI}

The most popular cloud-based Continuous Integration system at the moment is
Travis CI. Travis CI is written in the Ruby programming language and became
very popular because of the easy integration with the famous git-repository
hosting service GitHub\footnote{\url{https://github.com/}}.\\

Over 200.000 open-source projects like
nodeJS\footnote{\url{https://nodejs.org/}}, Ruby on Rails
\footnote{\url{http://rubyonrails.org/}} and emberJS
\footnote{\url{http://emberjs.com/}} and companies like
Facebook\footnote{\url{https://www.facebook.com/}},
Mozilla\footnote{\url{https://www.mozilla.org}} or
Twitter\footnote{\url{https://twitter.com/}} are using Travis CI as their
Continuous Integration system.\\ The configuration for Travis CI is done in one
simple YAML file with the name \lstinline|travis.yml| which must be located in
the root directory of a GitHub repository. There are two possible workflows how
you can integrate Travis CI:

\begin{itemize}
    \item Branch build flow
    \item Pull request build flow
\end{itemize}

\paragraph{Branch build flow} In the branch build flow GitHub automatically
triggers a Travis CI build every time you push a commit to your remote GitHub
repository.\\ If the build is OK and all the tests succeed, Travis CI is able to
deploy the code on a PaaS like Heroku\footnote{\url{https://www.heroku.com/}}.

\paragraph{Pull request build flow} In the pull request build flow everytime you
create a pull request on GitHub a Travis CI build is triggered.

\subsubsection{Drone}

Drone is developed by Google and like Travis CI a cloud-based Continuous
Integration system.\\

Drone integrates seamlessly with the two popular
version-control hosting services GitHub and Bitbucket
\footnote{\url{https://bitbucket.org/}} as well as Platform-as-a-service
providers such as Heroku, Dotcloud \footnote{\url{https://www.dotcloud.com/}},
Google AppEngine  \footnote{\url{https://cloud.google.com/appengine/}}.

