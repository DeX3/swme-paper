\section{Tooling}\label{sec:tooling}

There exists many good solutions for hosting a continious-integration System. 
This chapter handles the most popular ones.
We categorized the continious-integration solution in 2 parts.
The first part is about continious-integration-Systems, which are hosted on a own private server. 
The second part of this chapter is about continious-integration-Systems, which are hosted in the cloud, which are becoming increasingly popular in the open-source community.

\subsection{Self-hosted}\label{sec:tooling-self-hosted}

Self-hosted continious-integration systems are older than cloud-based continious-integration systems.
The software for the continious-integration-System runs localy on a private server 
and the users can reach and use this system over the network. 
The most popular solutions for a self-hosted continious-integration are:
\begin{itemize} 
    \item Jenkins(former Hudson) \footnote{\url{https://jenkins-ci.org/}}
    \item TeamCity \footnote{\url{https://www.jetbrains.com/teamcity/}}
    \item CruiseControl \footnote{\url{http://cruisecontrol.sourceforge.net/}}
\end{itemize}
The by far most remarkable and famoust one of this continious-integration-systems is Jenkins. 
Jenkins is written in the Java programming language and completly open source released under the MIT License.
Many top companies like netflix a\footnote{\url{https://www.netflix.com}}nd open source projects are using Jenkins as their continious-integration-system.
Jenkins runs in a servlet container on an application server like GlassFish \footnote{\url{https://glassfish.java.net/}}
or Apache Tomcat \footnote{\url{http://tomcat.apache.org/}}
on all common operating-systems.
There are many version-control tools included in Jenkins but the most relevant ones are git and svn.
The most powerfull feature of Jenkins is the extensibility of this continious-integration-System with plugins.
There are more than 1000 official plugins available for Jenkins to customize the continious-integration-System.  

\subsection{Cloud-based}\label{sec:tooling-cloud-based}

Cloud-based continious-integration systems are continious-integration-systems which are hosted in the cloud.
They are easier to configure than self-hosted continious-integration-systems but have their limitations in the extensibility.
They often have a monthly subscription pricing model for private version-control repositories 
and a free model for public and open-source version-control repositories.
Thats why they are becoming increasingly popular in the open-source community.
The most popular cloud-based continious-integrations are:
\begin{itemize} 
    \item Travis continious-integration \footnote{\url{https://travis-ci.org/}}

    \item Codeship \footnote{\url{https://codeship.com/}}

    \item Drone \footnote{\url{https://drone.io/}}

\end{itemize}
The most popular cloud-based continious-integration system at the moment is Travis continious-integration.
Travis continious-integration is written in the Ruby programming language and became very popular because of the easy integration with the famous git-repository hosting service GitHub.\footnote{\url{https://github.com/}}

Over 200.000 open-source projects like nodeJs,\footnote{\url{https://nodejs.org/}}
 Ruby on Rails \footnote{\url{http://rubyonrails.org/}}
and emberJs \footnote{\url{http://emberjs.com/}}

and companies like facebook,\footnote{\url{https://www.facebook.com/}}
 mozilla,\footnote{\url{https://www.mozilla.org}}
 twitter \footnote{\url{https://twitter.com/}}
are using Travis Ci as their continious-integration system.
The configuration for Travis Ci is done in one simple YAML file with the name travis.yml 
which must be located in the root directory of a GitHub repository.
There are two possible workflows how you can integrate Travis Ci:
\begin{itemize} 
    \item Branch build flow
    \item Pull request build flow
\end{itemize}
In the branch build flow GitHub automaticly triggers a Travis Ci build everytime you push a commit to your remote GitHub repository.
If the build is OK and all the tests succeed Travis Ci is able to deploy the code on a Paas like heroku.
In the pull request build flow everytime you create a pull request on GitHub a Travis Ci build is triggered.

